% chapters/06_conclusion.tex
\section{Summary}\label{sec:summary}

This thesis treats tensor-network workloads from an HPC perspective:
identify kernel bottlenecks on A100-class GPUs, apply profile-guided
optimisations, and quantify performance relative to vendor-library baselines.
The main control variables are launch overhead, memory-hierarchy efficiency,
and occupancy-resource trade-offs.

\section{Limitations}\label{sec:limitations}

The current scope has three deliberate limitations:

\begin{itemize}
  \item physics-specific algorithm derivations and notation are out of scope,
  \item optimisation focus is on Ampere-era kernels and one-node execution,
  \item coverage of workload diversity is bounded by available benchmark traces.
\end{itemize}

\section{Future Work}\label{sec:future-work}

Natural continuation points are:

\begin{enumerate}
  \item extending the same methodology to Hopper/H100-class hardware while
    preserving comparability with A100 results,
  \item deeper multi-GPU overlap strategies (communication/computation
    pipelining),
  \item automated kernel-configuration search constrained by profiler-derived
    bottleneck models.
\end{enumerate}
