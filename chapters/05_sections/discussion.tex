% chapters/05_sections/discussion.tex
\section{Discussion}\label{sec:discussion}

Kernel-level and system-level findings are combined here into practical guidance
for future optimisation cycles.

\subsection{What Improved and Why}

The main question is not only whether performance improved, but which mechanism
was responsible (launch reduction, memory locality, occupancy tuning, or
precision mode selection). Profiling evidence is used to support each claim,
using the fixed KPI pack defined in \cref{tab:kpi-pack} so that attributions
remain comparable across kernels and optimisation stages.

\subsection{How Well Results Transfer}

Improvements are classified as:

\begin{itemize}
  \item \textbf{likely to carry over} (expected to transfer across Ampere-like systems),
  \item \textbf{topology dependent} (depends on node NUMA/NVLink structure),
  \item \textbf{workload dependent} (effective only for the studied contraction
    distribution).
\end{itemize}

\subsection{Microbenchmark Implications for Tensor Networks}

The two most useful profiling microbenchmarks are the coalescing sweep
(\cref{tab:coalescing-sweep,fig:phenomena-coalescing}) and the register
pressure study (\cref{tab:register-pressure,fig:phenomena-register-pressure}).

For coalescing, the stride-1 to stride-32 degradation quantifies how quickly
memory efficiency collapses when warp lanes lose contiguous access patterns.
The result supports prioritising layout transforms and contraction ordering that
keep the hottest contracted index unit-stride in memory.

For register pressure, the occupancy drop from $100\%$ to $62.5\%$ with an
accompanying throughput loss shows that ``more in-register work'' is not
monotonically beneficial. In practice, this motivates tuning unroll depth and
accumulator count jointly with occupancy targets rather than maximising
instruction-level reuse in isolation.
