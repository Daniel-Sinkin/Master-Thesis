% chapters/03_sections/data_layout_and_memory_strategy.tex
\section{Data Layout and Memory Strategy}\label{sec:data-layout}

Data layout is treated as a first-class optimisation variable rather than a
post-processing detail. For contraction-heavy workloads, memory access patterns
often dominate arithmetic cost, so layout decisions are made jointly with
kernel design.

\subsection{Layout Principles}

The implementation follows four layout rules:

\begin{enumerate}
  \item keep the innermost thread-mapped dimension contiguous in memory to
    preserve coalesced global accesses,
  \item minimise explicit transpose and permutation kernels by selecting a
    stable canonical in-memory layout,
  \item perform reshape/pack operations in fused paths when possible,
  \item use alignment-friendly leading dimensions to improve transaction
    efficiency and vectorised loads.
\end{enumerate}

\begin{figure}[H]
  \centering
  \begin{tikzpicture}[
    >=Latex,
    every node/.style={font=\small, align=center},
    warp/.style={draw, rounded corners=2pt, fill=diagBlue, minimum width=4.0cm, minimum height=1.0cm},
    map/.style={draw, rounded corners=2pt, fill=diagGray, minimum width=4.6cm, minimum height=1.0cm},
    memgood/.style={draw, rounded corners=2pt, fill=diagGreen, minimum width=4.6cm, minimum height=1.0cm},
    membad/.style={draw, rounded corners=2pt, fill=diagRed, minimum width=4.6cm, minimum height=1.0cm},
    arrow/.style={->, very thick}
  ]
    \node[anchor=west, font=\bfseries] at (-0.1,1.7) {Coalesced (preferred)};
    \node[warp] (w1) at (1.9,1.0) {Warp lanes\\$i=0,\dots,31$};
    \node[map] (a1) at (7.2,1.0) {$\mathrm{addr}(i)=\mathrm{base}+4i$};
    \node[memgood] (m1) at (12.5,1.0) {Few aligned\\memory transactions};
    \draw[arrow] (w1) -- (a1);
    \draw[arrow] (a1) -- (m1);

    \node[anchor=west, font=\bfseries] at (-0.1,-0.1) {Strided (costly)};
    \node[warp] (w2) at (1.9,-0.8) {Warp lanes\\$i=0,\dots,31$};
    \node[map] (a2) at (7.2,-0.8) {$\mathrm{addr}(i)=\mathrm{base}+4si,\ s\gg1$};
    \node[membad] (m2) at (12.5,-0.8) {Many split\\memory transactions};
    \draw[arrow] (w2) -- (a2);
    \draw[arrow] (a2) -- (m2);

    \node[draw, rounded corners=2pt, fill=diagYellow, inner sep=4pt, align=center]
      at (7.2,-2.1) {Layout goal for this thesis: keep the thread-mapped inner
      dimension contiguous\\so that each warp issues mostly coalesced accesses.};
  \end{tikzpicture}
  \caption{Warp-level address mapping as a layout design rule: contiguous
  thread-to-address mapping improves global-memory transaction efficiency, while
  large strides fragment traffic.}
  \label{fig:coalesced-vs-strided-layout}
\end{figure}

\subsection{Memory-Level Strategy}

The memory strategy mirrors the A100 hierarchy:

\begin{itemize}
  \item \textbf{Registers:} accumulate partial results and keep loop-invariant
    scalars local.
  \item \textbf{Shared memory:} stage reused tiles and remove redundant global
    loads.
  \item \textbf{L2/HBM:} organise global accesses as contiguous bursts and avoid
    strided warp patterns unless unavoidable.
\end{itemize}

When tensors require non-trivial index reorderings, preprocessing overhead is
quantified explicitly and accounted for in end-to-end performance claims.
