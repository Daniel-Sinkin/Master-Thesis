% chapters/04_sections/kernel_design.tex
\section{Kernel Design}\label{sec:kernel-design}

Kernel design follows a bottom-up strategy: start from a minimal correct
implementation, then introduce one optimisation mechanism at a time (tiling,
fusion, vectorised loads, launch configuration tuning), validating each step by
profiling.

\subsection{Design Variants}

For each target contraction motif, three variants are maintained:

\begin{enumerate}
  \item \textbf{Reference variant:} straightforward kernel for correctness and
    baseline profiling.
  \item \textbf{Memory-optimised variant:} improved access locality and reduced
    redundant global loads.
  \item \textbf{Launch-optimised variant:} fused execution that amortises
    host-side launch overhead.
\end{enumerate}

Maintaining explicit variants makes performance deltas attributable and prevents
``all-at-once'' changes that are hard to analyse.

\subsection{Correctness and Numerical Checks}

Each variant is checked against a trusted baseline (library result or high
precision reference) with fixed tolerance settings per precision mode. Any
performance result is accepted only after numerical agreement is verified.

