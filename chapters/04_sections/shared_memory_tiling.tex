% chapters/04_sections/shared_memory_tiling.tex
\section{Shared Memory Tiling}\label{sec:tiling}

Shared-memory tiling is used to raise arithmetic intensity by reusing data that
would otherwise be fetched repeatedly from global memory. The design objective
is to maximise reuse while staying within per-block shared-memory limits and
avoiding bank conflicts.

\subsection{Tile Design Choices}

Tile sizes are selected based on:

\begin{itemize}
  \item contraction dimensions and divisibility,
  \item register pressure from accumulator blocking,
  \item shared-memory footprint per block,
  \item expected occupancy after resource allocation.
\end{itemize}

Padding is introduced where needed to avoid systematic bank conflicts in
column-wise shared-memory accesses.

\subsection{Boundary Handling}

Real workloads frequently use dimensions that are not multiples of tile sizes.
Boundary tiles are handled with predicated loads/stores to preserve correctness
without launching separate cleanup kernels.

