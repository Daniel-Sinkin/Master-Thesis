% chapters/04_sections/occupancy_and_launch_configuration.text
\section{Occupancy and Launch Configuration}\label{sec:occupancy}

Launch configuration is tuned as a constrained resource-allocation problem on
each SM. The central trade-off is that aggressive register/shared-memory usage
can increase per-thread efficiency while reducing occupancy, which may hurt
latency hiding.

\subsection{Tuning Procedure}

The tuning sequence is:

\begin{enumerate}
  \item choose an initial block size (typically 128 or 256 threads),
  \item inspect register use and shared-memory use per block,
  \item estimate achievable occupancy,
  \item benchmark and profile stall reasons,
  \item retune block size and launch bounds based on measured bottlenecks.
\end{enumerate}

If memory stalls dominate, higher occupancy is often preferred. If execution is
compute-bound with low dependency stalls, lower occupancy with higher
per-thread work can be superior.

\subsection{Launch Overhead Considerations}

For small kernels, launch latency can dominate irrespective of occupancy.
Therefore occupancy tuning is combined with launch reduction techniques (kernel
fusion or persistent execution) rather than treated in isolation.

